\documentclass{tufte-book}

%%% for no indentation at beginning of paragraph %%%
\makeatletter
% Paragraph indentation and separation for normal text
\renewcommand{\@tufte@reset@par}{%
	\setlength{\RaggedRightParindent}{0.0pc}%
	\setlength{\JustifyingParindent}{1.0pc}%
	\setlength{\parindent}{1pc}%
	\setlength{\parskip}{0pt}%
}
\@tufte@reset@par
% % % % % % % % % % % % % % % % % % % % % % % %

\geometry{textwidth=.6\paperwidth} % for smaller margin
\usepackage[utf8]{inputenc} % für Umlaute etc
\usepackage[ngerman]{babel} % nur damit funktionieren "` Anführungszeichen "'
\usepackage{hyperref} % für Links
\usepackage{enumitem} %for customising itemize, enumerate etc.
\setlist[itemize]{topsep=3pt,itemsep=2ex,partopsep=1ex,parsep=1ex}


\usepackage{amsmath}

\usepackage[color=teal!10,textsize=footnotesize, textwidth = 0.2\textwidth]{todonotes} % selbsterklärend
%\setuptodonotes{noline, color=teal!70}
%\newcommand{\done}[2]{\todo[#1, color=teal!10]{#2}}
\newcommand{\done}[2][]{\todo[color=teal!70, #1]{#2}}





\usepackage{datetime} % for date
\usepackage{lipsum}
\newcommand{\newday}[2]{\vspace{2cm}\marginnote{{\Large \textbf{\shortdayofweekname{#1}{#2}{2021}, ~\formatdate{#1}{#2}{2021}}}}}



\begin{document}
\newday{8}{4}
Neues Journal eingerichtet!
Übertrag der Todos aus dem alten MA-Journal:

\begin{itemize}
	\item Journal polishen: ToDo Package und Datum... \done{Journal}
	\item Rechte eingeschränkt? checken \done{Rechte}
	\item Koordinatenkram aufschreiben: Graphik der Koordinatensysteme (Neigung der Erd- und Sonnenachse?!), Ulysses Trajektorie Anwendung? \todo{Koordinatenkram aufschreiben}
	\item in calc\_d90 Division checken: python3-kompatibel? Schaltjahr...? \done{Schaltjahr}
	\item SPICE Skript überarbeiten \todo{Ulysses\_SPICE}
	\begin{itemize}
		\item überlegen, wie ich manuelle Rotation einbaue \todo{instance: rotation by hand}
		\item Winkel einbauen \todo{AspAngles}
		\item Funktion zum Wegschreiben der Trajektorie \todo{Write}
		\item Geschwindigkeiten alt/neu \todo{v}
		\item 3D-Repräsentation überlegen \todo{3D Plot}
	\end{itemize}
	\item Koordinatenskripte überarbeiten: etCoords ersetzen \todo{Koordinatenskript}
\end{itemize}
%Weiter: testen, wie gerade der Stand der Plots in ulysses\_spice ist.

\newday{9}{4}
\begin{itemize}
	\item in calc\_d90 Division checken: python3-kompatibel? Schaltjahr...?\\$\Rightarrow$ ersetzt durch //, dann passt es sowohl für Python2 als auch Python3. \\Jetzt noch überlegen, wie das mit den ersten zwei Monaten von Schaltjahren ist. Das passt doch nicht?! Aber evtl. egal, wenn das nur intern benutzt wird und das hier überall gleich gemacht wird...? \\
	$\Rightarrow$ Das ganz Schaltjahr selbst ist um einen Tag verschoben...\\
	Lösung: \\
	offd = offy*365 + (offy.astype(int)+1)//4 \\(statt +2)\done{Schaltjahr}
	\item Mit Verena versucht, svn+ssh den Password-Prompt abzugewöhnen. Auf dem Weg mitgenommen:
	\begin{itemize}
		\item \textit{hostname -I} zeigt IP an
		\item \textit{ip a} irgendwie auch, aber auch allerhand anderes...
		\item nach dem Bearbeiten der ~/.bashrc: \textit{source ~/.bashrc´} eingeben
		\item Lösung des Problems: Neues ssh-Schlüsselpaar erzeugen mit \\ \textit{ssh-keygen -t rsa -f ~/.ssh/newkey} (Name ist dann newkey)\\dann \textit{ssh-add /home/asterix/fischer/.ssh/newkey}\\ \textit{ssh-add -L} zeigt public keys des agents an \\ Dann noch den public key in .ssh/authorized\_keys kopieren
	\end{itemize}

\end{itemize}
\newday{25}{5}
Next: Rotation einbauen als Methoden in beiden Unterklassen, dafür Coordinates-Skript checken (hg\_to\_hc oder andersrum etc.), dann versuchen die Plots aus Vortrag zu reproduzieren. Falls Vererbung Quatsch ist: Kein Ding, kann ich leicht wieder zurück bauen!
\end{document}