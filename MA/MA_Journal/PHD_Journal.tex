\documentclass{tufte-book}

%%% for no indentation at beginning of paragraph %%%
\makeatletter
% Paragraph indentation and separation for normal text
\renewcommand{\@tufte@reset@par}{%
	\setlength{\RaggedRightParindent}{0.0pc}%
	\setlength{\JustifyingParindent}{1.0pc}%
	\setlength{\parindent}{1pc}%
	\setlength{\parskip}{0pt}%
}
\@tufte@reset@par
% % % % % % % % % % % % % % % % % % % % % % % %

\geometry{textwidth=.6\paperwidth} % for smaller margin
\usepackage[utf8]{inputenc} % für Umlaute etc
\usepackage[ngerman]{babel} % nur damit funktionieren "` Anführungszeichen "'
\usepackage{hyperref} % für Links
\usepackage{enumitem} %for customising itemize, enumerate etc.
\setlist[itemize]{topsep=3pt,itemsep=2ex,partopsep=1ex,parsep=1ex}


\usepackage{amsmath}

\usepackage[textsize=footnotesize, textwidth = 0.2\textwidth]{todonotes} % selbsterklärend
\setuptodonotes{noline, color=teal!70}
%\newcommand{\done}[2]{\todo[#1, color=teal!10]{#2}}
\newcommand{\done}[2][]{\todo[color=teal!10, #1]{#2}}





\usepackage{datetime} % for date
\usepackage{lipsum}
\newcommand{\newday}[2]{\vspace{2cm}\marginnote{{\Large \textbf{\shortdayofweekname{#1}{#2}{2021}, ~\formatdate{#1}{#2}{2021}}}}}



\begin{document}
\newday{8}{4}
Neues Journal eingerichtet!
Übertrag der Todos aus dem alten MA-Journal:

\begin{itemize}
	\item Journal polishen: ToDo Package und Datum... \done{Journal}
	\item Rechte eingeschränkt? checken \done{Rechte}
	\item Koordinatenkram aufschreiben: Graphik der Koordinatensysteme (Neigung der Erd- und Sonnenachse?!), Ulysses Trajektorie Anwendung? \todo{Koordinatenkram aufschreiben}
	\item in calc\_d90 Division checken: python3-kompatibel? Schaltjahr...? \todo{Schaltjahr}
	\item SPICE Skript überarbeiten \todo{Ulysses\_SPICE}
	\begin{itemize}
		\item überlegen, wie ich manuelle Rotation einbaue \todo{instance: rotation by hand}
		\item Winkel einbauen \todo{AspAngles}
		\item 3D-Repräsentation überlegen \todo{3D Plot}
	\end{itemize}
	\item Koordinatenskripte überarbeiten: etCoords ersetzen \todo{Koordinatenskript}
\end{itemize}
Weiter: testen, wie gerade der Stand der Plots in ulysses\_spice ist.

\end{document}