% Chapter Template

\chapter{Outlook and Conclusion} % Main chapter title

\label{chap:concl} % Change X to a consecutive number; for referencing this chapter elsewhere, use \ref{ChapterX}



The aim of this work was to resolve the VDF of He+ PUI in three dimensions.
In a first step He+ data was selected from the raw Ulysses SWICS PHA data. He+ events were identified by analyzing SWICS' combined measurement of energy-per-charge, time-of-flight and residual energy.


For resolving a three dimensional velocity bla SWICS' detector and sector informations were considered together with the orientation and eigen-velocity of Ulysses. In particular, 



We utilized a virtual detector to combine the da

\\
Mithilfe des virtuellen Detektors können jetzt geeignete (Triples) PHA Worte in dreidimensionale Geschwindigkeitskomponenten übersetzt werden (vorher: nur Betrag. Jetzt: Komponenten).
\\
Versucht den Kollimator für Ulysses SWICS nachzubauen mit all seinen Feinheiten (Aspect Winkel, Sunpulser, Trajektorie, Geometrie, Eigengeschwindigkeit...)



The thesis starts with an introduction into PUI, before the Ulysses mission and the instrument SWICS with its measurement principle are described. 
Coordinate Systems are described
Virtual detector is constructed 
Consider orientation and eigen-velocity of the instrument
Tested Virtual Detector


Auflösung eingeschränkt
Wir sind eingeschränkt auf schnellen SW


aim: directional resolution of velocity of incident He ions
SWICS data in its entirety
together with
Ulysses position and orientation 

\\
Für He+ durchgeführt weil PUIs dankbar mit breiter Verteilung für diese grobe Richtungsauflösung und He das häufigste PUI. Könnte man aber auch für andere PUIs machen! (he2+, O6+,...)



\section{Outlook}
\begin{itemize}
	\item Efficiency müsste genau bestimmt werden: Dabei berücksichtigen, welcher Anteil unter den Threshold wandert. Evtl. überschätzen wir die Eff., wenn wir die interpolierten Werte von ACE/SWICS nehmen.
	\item B-Feld-spezifische Untersuchung: Torus
	\item Radialabhängigkeit: adiabatic Cooling, PA-Scattering
	\item Sonnenwindanalyse: Ist zu breit. In Richtung Spin-Achse könnte man aber über die Schalen analysieren (Lars fragen: radial?)
	
\end{itemize}

