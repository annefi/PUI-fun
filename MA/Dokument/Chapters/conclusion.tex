% Chapter Template

\chapter{Outlook and Conclusion} % Main chapter title

\label{chap:concl} % Change X to a consecutive number; for referencing this chapter elsewhere, use \ref{ChapterX}


\section{Outlook}
\begin{itemize}
	\item Efficiency müsste genau bestimmt werden: Dabei berücksichtigen, welcher Anteil unter den Threshold wandert. Evtl. überschätzen wir die Eff., wenn wir die interpolierten Werte von ACE/SWICS nehmen.
	\item B-Feld-spezifische Untersuchung: Torus
	\item Radialabhängigkeit: adiabatic Cooling, PA-Scattering
	\item Sonnenwindanalyse: Ist zu breit. In Richtung Spin-Achse könnte man aber über die Schalen analysieren (Lars fragen: radial?)
	
\end{itemize}




aim: directional resolution of velocity of incident He ions
SWICS data in its entirety
together with
Ulysses position and orientation 
\\
Mithilfe des virtuellen Detektors können jetzt geeignete (Triples) PHA Worte in dreidimensionale Geschwindigkeitskomponenten übersetzt werden (vorher: nur Betrag. Jetzt: Komponenten).
\\
Versucht den Kollimator für Ulysses SWICS nachzubauen mit all seinen Feinheiten (Aspect Winkel, Sunpulser, Trajektorie, Geometrie, Eigengeschwindigkeit...)
\\
Für He+ durchgeführt weil PUIs dankbar mit breiter Verteilung für diese grobe Richtungsauflösung und He das häufigste PUI. Könnte man aber auch für andere PUIs machen! (he2+, O6+,...)