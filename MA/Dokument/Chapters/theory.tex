% Chapter 1

\chapter{Pickup Ions} % Main chapter title

\label{Chapter1} % For referencing the chapter elsewhere, use \ref{Chapter1} 

%-------------------------------------------------------------



%-------------------------------------------------------------

Pickup ions are created when neutral atoms inside the heliosphere become ionised and are subsequently swept away with the heliospheric magnetic field that is embedded within the solar wind.


\section{The Heliosphere}

Oder Überkapitel Solar Physics?
\\ \\
Heliosphere: Grenze zu LISM
\\ \\
Solar Wind: Zusammensetzung, schneller und langsamer \\ \\
B-Feldgleichung

%%%
%-------------------------------------------------------------
%%%

\section{Pickup Ions}
A neutral atom inside the heliosphere is only subjected to the gravitational force and radiation pressure of the sun. It is not sensitive to any electromagnetic forces until it becomes ionised by solar ultra-violet radiation, charge exchange with solar wind protons or electron impact (Q?). After ionisation the particle starts interacting with the solar wind plasma. In particular it is forced onto a gyro orbit about the heliospheric magnetic field
that is embedded within the solar wind. As the freshly created ion is swept away with the magnetic field line it is \say{picked up} from its location of ionisation -- a new pickup ion (PUI) has been created.
\\ \\
PUIs were first observed by \citet{moebius_nature_85} with the SULEICA Instrument on the AMPTE spacecraft. The particles measured at $1\,\mathrm{AU}$ were He+ ions of interstellar origin.
\\ \\
Once the particle is ionised, its probability to become ionised another time decreases (Quelle). This characteristic of being only singly charged can help to discriminate PUIs from solar wind ions, that are mostly more often charged (Q?).
\\ \\
PUIs are mostly only single charged. This characteristic can help to distinguish them from solar wind ions of coronal origin which often have been ionized multiple times, if not completely. (Q?)
\\
VDF non-maxwellian, spatial density pattern 
\\ \\
There have been observed several species of PUIs: 
\section{Interstellar Pickup Ions}
Heliosheath, relative motion \\ \\
The neutral part of the LISM can enter the heliosphere as it is not affected by the heliosheath (Todo). Inside the heliosphere the neutrals are guided only by the gravitational force and radiation pressure of the sun. The neutral particle's species determines how deep it can travel into the heliosphere before it becomes ionized. Species with a higher First Ionization Potential will be able to approach the sun much closer without being ionized. This results in He+ being the dominant PUI species at a solar distance of $1\,\mathrm{AU}$ even if in the LISM the abundance of hydrogen is about 10 times the one of helium.
\begin{itemize}
	\item ionisation process is also dependent of the species
	\item Spatial distribution:\\
	gravitational force and radiation pressure lead to two regions of enhanced density of neutrals (in the ecliptic): Focusing cone and crescent.
	Focusing cone: For species with high FIP (as the others are ionized before and do not reach the downwind side of the sun)
	\item variation of He+ with the solar cycle: Rucinski 2003
	\item H, O and N are depleted in the filtration region (Baranov Malama 1995), Wimmer Skript: even before ioniztion: density is determined by ratio of gravitational force and photon pressure
	\item neutral density determines PUI production rate
\end{itemize}

\section{Inner-source Pickup Ions}

\section{VDF}
After the particle has been ionised it is forced onto a gyro motion about the local field line of the heliospheric magnetic field due to the Lorentz force. 
\\ \\
No velocity compared to vsw
\\ \\
B field is convected by the SW and moves radially outwards from the sun
\\ \\
Form of the torus depends on several factors: velocity of the particle before the PU process, orientation of the magnetic field