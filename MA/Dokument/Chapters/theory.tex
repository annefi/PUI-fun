% Chapter 1

\chapter{Pickup Ions} % Main chapter title

\label{Chapter1} % For referencing the chapter elsewhere, use \ref{Chapter1} 

%-------------------------------------------------------------



%-------------------------------------------------------------

Pickup ions are created when neutral atoms inside the heliosphere become ionised and are subsequently swept away with the heliospheric magnetic field that is embedded within the solar wind.


\section{The Heliosphere / Introduction}

Oder Überkapitel Solar Physics?
\\ \\
Heliosphere: Grenze zu LISM
\\ \\
Solar Wind: Zusammensetzung, schneller und langsamer, high latitudes: less complex, constant in speed \\ \\
B-Feldgleichung
\\ \\
Irgendwohin muss unbedingt Motivation, warum PUIs überhaupt interessant sind zu messen

%%%
%-------------------------------------------------------------
%%%

\section{Pickup Ions}
A neutral atom inside the heliosphere is only subjected to the gravitational force and radiation pressure of the sun. It is not sensitive to any electromagnetic forces until it becomes ionised by solar ultra-violet radiation, charge exchange with solar wind protons or electron impact (Q?). After ionisation the particle starts interacting with the solar wind plasma. In particular it is forced onto a gyro orbit about the heliospheric magnetic field
that is embedded within the solar wind. As the freshly created ion is swept away with the magnetic field line it is \say{picked up} from its location of ionisation -- a new pickup ion (PUI) has been created.
\\ \\
PUIs were first observed by \citet{moebius_nature_85} with the SULEICA Instrument on the AMPTE spacecraft. The particles measured at $1\,\mathrm{AU}$ were He+ ions of interstellar origin.
\\ \\
Once the particle is ionised, its probability to become ionised another time decreases (Quelle). This characteristic of being only singly charged can help to discriminate PUIs from solar wind ions, that are mostly more often charged (Q?).
\\ \\
PUIs are mostly only single charged. This characteristic can help to distinguish them from solar wind ions of coronal origin which often have been ionized multiple times, if not completely. (Q?)
\\
VDF non-maxwellian, spatial density pattern 
\\ \\
There have been observed several species of PUIs: 
\section{Interstellar Pickup Ions}
Heliosheath, relative motion \\ \\
The neutral part of the LISM can enter the heliosphere as it is not affected by the heliosheath (Todo). Inside the heliosphere the neutrals are guided only by the gravitational force and raOriginofC-Geissdiation pressure of the sun. The neutral particle's species determines how deep it can travel into the heliosphere before it becomes ionized. Species with a higher First Ionization Potential will be able to approach the sun much closer without being ionized. This results in He+ being the dominant PUI species at a solar distance of $1\,\mathrm{AU}$ even if in the LISM the abundance of hydrogen is about 10 times the one of helium.
\begin{itemize}
	\item ionisation process is also dependent on the species
	\item radiation pressure only important for H (and He?). Kepler orbit...
	\item Spatial distribution:\\
	gravitational force and radiation pressure lead to two regions of enhanced density of neutrals (in the ecliptic): Focusing cone and crescent.
	Focusing cone: For species with high FIP (as the others are ionized before and do not reach the downwind side of the sun)
	\item variation of He+ with the solar cycle: Rucinski 2003
	\item H, O and N are depleted in the filtration region (Baranov Malama 1995), Wimmer Skript: even before ioniztion: density is determined by ratio of gravitational force and photon pressure
	\item neutral density determines PUI production rate
\end{itemize}

\section{Inner-source Pickup Ions}
The idea of an additional source for the PUI's neutral seed population was born when \citet{geiss_1995a} measured a global distribution of C+ PUIs with the SWICS instrument on Ulysses. 
Interstellar carbon exists almost exclusively in a single charged state \citep{Frisch} in the LISM. As only neutral atoms can enter the heliosphere it was not expected to find a distinct signature of C+ pickup ions.
However, pickup carbon was observed with about the same ratio as oxygen, of which, in contrast, ~80\% is in a neutral charge state in the interstellar medium.
These findings suggested that there must be another source for neutrals that has its origin somewhere inside the heliosphere. \\
In following studies \citep[e.g.][]{geiss_1995b} there were found also other species like O+ and Ne+ of these, so called, inner-source PUIs.
\\
Inner-source PUIs show a composition that is similar to the one of the solar wind \citep{gloeckler2000_innersource, allegrini_2005} as well as a velocity distribution function that is centered around $w_{SC} \approx 1$ \citep{schwadron_2000} and seems to have thermalized with the solar wind.
\\
Beneath those two characteristics there are other aspects concerning inner-source PUIs, that are still under debate. In particular that is the production mechanism of their neutral seed population.
\citet{allegrini_2005} has summarized current candidates for possible scenarios. Two of those give an explanation for the ion's composition as they directly incorporate solar wind ions in the process:
\begin{itemize}
	\item Solar wind recycling \citep{gloeckler2000_innersource, schwadron_2000}: Absorption of solar wind ions by heliospheric grains and subsequent reemission of neutral atoms
	\item Solar wind neutralization \citep{wimmer_2002}: Solar wind ions penetrate sub-micron-sized dust grains and undergo (partial) neutralization by charge exchange
\end{itemize}
(As this work does not focus on inner-source PUIs in particular...)


\section{VDF}
After the particle has been ionised it is forced onto a gyro motion about the local field line of the heliospheric magnetic field due to the Lorentz force. 
\\ \\
To examine the velocity distribution of PUIs after they have been ionized we need to consider the initial speed $v_{ini}$ of the neutral particle. 
For neutrals from the LISM this is mainly given by the inflow speed $v_{ISM}$ of the local interstellar medium with which they enter the heliosphere. As we don't exactly know about the production mechanism of inner-source PUIs, the following considerations mainly relate to interstellar PUIs.
\citet{schwadron_2015_ibex} obtained $v_{ISM} \approx 25 \,\mathrm{km\,s^{-1}}$ with the IBEX satellite for helium. Considering the acceleration by the sun's gravitational force we have a maximum initial speed of $v_ini \approx 50 \,\mathrm{km\,s^{-1}}$ at $1\,\mathrm{AU}$. Compared to an average solar wind speed of $v_{sw} \approx 400\,\mathrm{km\,s^{-1}}$ one can neglect this initial speed in a first step.

For simplicity we thus consider a neutral particle at rest that becomes ionized by one of the aforementioned processes. The freshly created ion now is subjected to the electromagnetic forces of the solar wind plasma. In particular, it finds itself at a velocity $v_{sw}$ relative to the magnetic field which is convected outwards by the solar wind that is assumed to flow radially outwards. Due to the Lorentz force the PUI starts to gyrate about the magnetic field line on an orbit that is perpendicular to it. \\
When we further consider a magnetic field's orientation that is perpendicular to the solar wind flow, $\vec{B} \perp \vec{v}_{sw}$, the ion's gyration speed is $v_{sw}$ while its guiding center moves together with the field line at a speed of $v_{sw}$ as well.
Thus, the total speed of the PUI ranges between $0\, \mathrm{v_{sw}} $ and $2 \, \mathrm{v_{sw}}$ in a sun frame of reference. \\
As the heliospheric magnetic field lines are shaped like an Archimedean spiral, the so called \textit{Parker spiral}, the assumption of a perpendicular magnetic field only applies when solar wind speed $v_{sw}$ and solar distance $r_\odot$ follow the relation
\begin{align*}
90 ^\circ \approx  arctan \left( \frac{2\pi}{T_\odot \cdot v_{sw}} r_\odot \right)
\end{align*}
with sun's sidereal period $T_\odot \approx 25\,d$ \citep{prlss_2004}.
In other cases, e.g. for solar distances about $1\,\mathrm{AU}$, at which the angle between solar wind and magnet field direction is approximately $45^\circ$, the maximum speed in a sun frame of reference is decreased. In general, the gyration speed is given by
\begin{align*}
todo
\end{align*}
with ... .
\\ \\
The velocity space for a pickup situation with a non-radial magnetic field orientation is shown in figure todo on the right.
The PUI's total velocity consists of the movement of the guiding center (...) and the gyration velocity (...). We note, that in this case there is a relative velocity between the motion of the solar wind bulk and the PUI's guiding center movement. 
\\
However, independent on the magnetic field orientation, every possible velocity space trajectory is part of a sphere with the radius $v_{sw}$ centered around $\vec{v}_{sw}$. That means that, in the frame of the solar wind, the freshly created PUI always moves with a speed that is as fast as the solar wind itself. (todo: Hier w einführen?) 
\\ \\
Instead of a single PUI we can consider an ensemble of PUI's that is injected into the solar wind while the magnetic field orientation is not changing much. For that we expect the VDF to form a ring shape in velocity space, commonly called the "PUI torus VDF" \citep{oka_2002}.
The expected orientation of this highly anisotropic torus VDF depends on the local magnetic field direction and is sketched in figure todo for three different angles.
\\ \\
...thickness that is related (associated) to the neutral's velocity and is very small compared to the radius...\\ \\
Spatial diffusion: chalov Fahr 1998. Signature of plasma parcel in which is was produced doesn't match with the one it is measured in
\\ \\
After the injection, the PUI population is radially carried away with the solar wind. During phase space transport through the heliosphere the PUIs are subjected to multiple processes that are expected to modify the shape of the initial toroidal VDF. However, it is not completely understood how the VDF evolves in detail.
\\ \\
A fast isotropization of the VDF due to pitch-angle scattering was suggested by \citet{vasyl_siscoe_1976} in a theoretical work.
However, observations by e.g. \citet{moebius_98} on $He+$ or \citet{gloeckler_1995} on TODO have shown clear anisotropic features in the measured VDFs.
Following studies (todo) explained these findings with the assumption that the ions would be injected into the sunward hemisphere of velocity space more likely. Ineffective pitch-angle scattering into the antisunward hemisphere thus would result in an radial anisotropy.
\\ \\
Recent observations have emphasized the influence of the magnetic field direction on the measured anisotropy.
Utilising 2D analyses of the velocity space, \citet{oka_2002} and \citet{drews_2015} found that the measured VDF of PUIs is systematically oriented about the direction that is perpendicular to the magnetic field.
Thus, it is believed that the VDF's anisotropic features are remnants of the initial toroidal VDF. (and the pa scattering didnt have enough time to isotropize the distribution)
\\ \\
Furthermore, there are different acceleration and deceleration processes that change the PUI's initial VDF and lead to a diffusion in velocity space.
Under the assumption of an isotropic VDF the PUI population is often treated as an adiabatic gas that is consequently cooled when expanding with the solar wind. This picture, initially suggested by \citet{vasyl_siscoe_1976}, however, must be reviewed due to the doubtful fact of a fully isotropic VDF.
Another cooling mechanism, called the \textit{magnetic cooling}, is due to the magnetic field weakening with solar distance. As the PUIs are swept outwards both their ... and their ...invariant have to be conserved which leads to a decrease in both velocity components (parallel and perpendicular to the magnetic field) and thus to a decrease in total velocity (in the frame of the solar wind).
\\ \\
(focusing (adiabatic invariant) \& Ginzberg Landau (Fahr2008): "magnetic cooling" (auch gute Erklärung: Fahr\&Fichtner2011))
\\ \\
Acceleration of PUIs can be caused by 
acceleration: first and seconf order fermi (verstehen, gründe): eher außen bzw. ehr innen. Außerdem ein Mechanismus, der nicht an einzelne Events gebunden ist, sondern immer vorhanden: Mechanismus für alle Teilchen, power law -5...\\
man kann in der 1D Verteilung beobachten, dass 2vsw exceeded wird
\\ \\
PUI He should be measured throughout the mission as they penetrate the heliosphere until 0.5 AU \citep{gloeckler_1992}
\\ \\ \\
Instrument that is capable of measuring this distribution: large acceptance in absolute velocity, large variation and resolution in angles
\subsection{1D reduced VDF, aim of this work...?}