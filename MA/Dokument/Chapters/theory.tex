% Chapter 1

\chapter{Theoretical Background} % Main chapter title

\label{Chapter1} % For referencing the chapter elsewhere, use \ref{Chapter1} 

%-------------------------------------------------------------



%-------------------------------------------------------------

\section{Solar Wind}

Oder Überkapitel Solar Physics?

Heliopshere

%%%
%-------------------------------------------------------------
%%%

\section{Pickup Ions}
Pickup ions are created when neutral atoms within the heliosphere become ionised and are subsequently \say{picked up} by the surrounding solar wind plasma. 
\\ \\
Neutral particles inside the heliosphere are only subjected to gravitational force and radiation pressure of the sun. 
\\ \\
Ionisation by photoionisation by solar ultra-violet radiation, charge exchange or electron impact (Quelle?)
\\ \\
After ionisation the particles start interacting with the solar wind plasma. In particular they are forced onto gyro orbits around the local magnetic field (that is embedded in the solar wind and swept away outwards) due to the Lorentz force.
\\ \\
First described by... name given...
\\ \\
Two sources of neutral atoms: Interstellar and Inner source
\\ \\
Velocity distribution
\\ \\
Discriminate from solar wind: VDF non-maxwellian and mostly single charged.
Once the particle is ionised, its probability to become ionised another time decreases (Quelle). This characteristic of being only singly charged can help to discriminate PUIs from solar wind ions, that are mostly more often charged (Q?).

\subsection{VDF}
