% Chapter 1

\chapter{Pickup Ions} % Main chapter title

\label{Chapter1} % For referencing the chapter elsewhere, use \ref{Chapter1} 

%-------------------------------------------------------------



%-------------------------------------------------------------

Pickup ions are created when neutral atoms inside the heliosphere become ionised and are subsequently swept away with the heliospheric magnetic field that is embedded within the solar wind.


\section{The Heliosphere}

Oder Überkapitel Solar Physics?
\\ \\
Heliosphere: Grenze zu LISM
\\ \\
Solar Wind: Zusammensetzung, schneller und langsamer \\ \\
B-Feldgleichung

%%%
%-------------------------------------------------------------
%%%

\section{Pickup Ions}
A neutral atom inside the heliosphere is only subjected to the gravitational force and radiation pressure of the sun. It is not sensitive to any electromagnetic forces until it becomes ionised by solar ultra-violet radiation, charge exchange with solar wind protons or electron impact (Q?). After ionisation the particle starts interacting with the solar wind plasma. In particular it is forced onto a gyro orbit about the heliospheric magnetic field
that is embedded within the solar wind. As the freshly created ion is swept away with the magnetic field line it is \say{picked up} from its location of ionisation -- a new pickup ion has been created.
\\ \\
First described by... name given...
First measured by Moebius, He+
\\ \\
Warum gerade He+ untersuchen? Häufigstes at 1 AU
\\ \\
Discriminate from solar wind: VDF non-maxwellian and mostly single charged.
Once the particle is ionised, its probability to become ionised another time decreases (Quelle). This characteristic of being only singly charged can help to discriminate PUIs from solar wind ions, that are mostly more often charged (Q?).
\\ \\
Two sources of neutral atoms: Interstellar and Inner source
 Häufigkeit spiegelt nicht LISM-Häufigkeit wider wegen FIP
\section{VDF}
After the particle has been ionised it is forced onto a gyro motion about the local field line of the heliospheric magnetic field due to the Lorentz force. 

No velocity compared to vsw
