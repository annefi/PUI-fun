% Chapter Template

\chapter{Instrumentation} % Main chapter title

\label{ChapterInstrumentation} 

%----------------------------------------------------------------------------------------

%----------------------------------------------------------------------------------------

\section{Ulysses}
\label{sec:ulysses}
The Ulysses spacecraft was launched in 1990 and orbited the sun for nearly 20 years as a joint ESA/NASA project.
Ulysses' most remarkable feature is its out-of-ecliptic orbit with an maximum inclination of TODO.
As the first spacecraft it was hence capable of taking in-situ measurements from above the poles of the sun.
\\ \\
The original main scientific objectives of the mission were:
\begin{itemize}
	\item solar wind, heliospheric magnetic field, solar radio bursts, plasma waves, solar X-rays, solar and galactic cosmic rays, interstellar and interplanetary neutral gas and dust with respect to the heliospheric latitude
	\item Jupiter flyby: Jupiter's magnetosphere
	\item search for gravitational waves
	\item detection of cosmic gamma ray bursts
\end{itemize}
For this aim Ulysses was equipped with a wide range of different instruments and antennas. One of the in-situ instruments is the Solar Wind Ion Composition Spectrometer (SWICS), that will be described in the next chapter.\\
Ulysses was launched in October 1990 and left earth's gravitational field with $15.4\,\mathrm{km/s}$. Starting with a flyby maneuver around Jupiter Ulysses was sent onto its highly elliptical orbit.
With an orbital period of 6.2 years Ulysses completed nearly three orbits around the sun until communication was shut down in June 2009 due to the expiring of the radioisotope thermal generators.
Within the mission's long lifetime the Sun's behaviour over its activity cycle of 22 years could be studied. 
\\ \\
scientific payload  -- trajectory: Jupiter, aphelion, perihelion -- aspect angle (antenna)-- Power: RHU because too far away for solar panels (and radiation belt of Jupiter) -- rotation stabilization -- long time: solar periods -- mission was extended multiple times
\\ \\
Abbildung Orbit, Abbildung Sun Cycles?
\\ \\ 
Extra Kapitel Orbit? \\
Perihel, Aphel, spinstabilisiert, Antenna fast Rotationsachse, Antenne zeigt zur Erde, woher Daten (Ulysses, SWICS, Erde)
\\ \\
Rotation, Antenna
\\ \\ 
Importance PUIs: ref to PUI-Kapitel
%
%
%
%
%
\section{SWICS}

\subsection{Introduction and Objectives}
The Solar Wind Ion Composition Spectrometer (SWICS, \citet{gloeckler_1992}) is a time-of-flight mass spectrometer mounted on the spacecraft Ulysses (s. section \ref{sec:ulysses}). The instrument is designed to determine the elemental and charge-state composition and the velocity distribution of solar wind ions. With an energy-per-charge range from $0.16 \, \mathrm{keV/e}$ to $59.6 \, \mathrm{keV / e}$ SWICS is in principle able to measure every solar wind ion species from protons to iron with any typical charge state. Depending on the individual ion, energies from $E < 1 \,\mathrm{keV}$ up to $E > 1 \, \mathrm{MeV}$ are covered.
\\ \\
Additionally, the flight spare of SWICS has been mounted on the spacecraft ACE (TODO: ref Stone 1998)
\\ \\ 
Also capable of measuring PUIs! Erstmals gesehen...?
\\ \\
SWICS measures the mass $m$, the charge $q$ and the energy $E_{SSD}$ of entering ions by a combination of three separate measurements: The \textit{electrostatic deflection analyzer} within SWICS entrance systems is used for determining the energy per charge of a particle. Within the time-of-flight/energy section the particle's time-of-flight ($ToF$) and energy ($E_{SSD}$) are measured. \\ 
Todo: Bild von SWICS, neben den beiden sieht man noch die Elektronik...
\\ \\
In the following section the measurement is described in more detail.
\subsection{Principle of Measurement / Identification of Particles}
\subsubsection{Collimator and Electrostatic Analyzer}

\subsubsection{Time-of-flight measurement}

\subsubsection{Energy measurement}

\subsection{Velocity Space Coverage}