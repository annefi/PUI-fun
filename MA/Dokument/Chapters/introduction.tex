% Chapter Template

\chapter{Introduction} % Main chapter title

\label{chap:intro} % Change X to a consecutive number; for referencing this chapter elsewhere, use \ref{ChapterX}


kurze Übersicht:
was ist das Problem (Fragestellung), wie löse ich das, in dieser Arbeit wird das und das gemacht
(A. etwas ausführlicher wdh.)

Der Phasenraumtransport von PUI wurde bereits sehr detailliert modelliert. Bestätigung durch Messungen hängt aber zurück, da die Messungen häufig auf 1D reduziert sind.
In dieser Arbeit soll ein Tool erstellt werden, das 3d auflösen kann auf Basis von Daten, die bisher nicht ausreichen ausgewertet wurden
Tool bauen anhand for Ulysses/SWICS Daten. SWICS hat grobe Richtungsauflösung (Winkel angeben). 



Warum PUI?
weil breite Verteilung. Ideale Population, um Werkzeuge zu tsten und anzuweneden
(weil räumliche Auflösung halt doch nicht so toll ist: Winkelschritte von...)

PUI differ from solar wind ions by a very non-Maxwellian Velocity Distribution Function. As this distribution is particularly broad, PUI are the perfect candidates for testing the METHOD and being studied with it.



Ulysses
weil viele PUI Studioen darauf aufgebaut haben
kann radiale Entwicklung auflösen




Motivation:
Gloeckler zeigen
Ohne Modellvorstellung kann man anhand dieses Spektrums...
wurde nicht aus einer eindeutigen Verteilung erstellt

Es gibt unendlich viele 3D-Modellfunktionen, die auf 1D reduziert so eine Kurve darstellen (Uneindeutigkeit der reduzierten Daten). Man verliert Informationen. Darum macht es keinen Sinn, 1D zu betrachten, wenn man die Möglichkeit hat, sich 3D anzuschauen. 

***
Zentrale Frage: Phasenraumtransport verstehen
Und die gemachten Modelle, z:B auch Kühlen, verstehen.
9) Trafo
Logisch: Bezugsframe ist der Sonnenwind und kein künstlicher SC-Frame, d.h. wir müssen übergehen. Dafür brauchen wir aber die 3D Informationen

...wie würde das in 3D aussehen?

“Drews 2013: While the assumption of a fully isotropic He+ VDF would allow for the transition from the SC to a SW frame of reference even without knowing the three components of the He+ velocity vector, [...] a non isotropic He+ VDF [...] will result in a significant difference in shape and intensity of the observed spectra between the two frames of refrence,
Man kann in 1D nicht die einzelnen Prozesse unterscheiden, die zu einer bestimmten Verteilung führen
Warum sw frame?\\
Warum brauchen wir 3D? Um das zu entfalten und dann den Frame wechseln zu können.





Gliederung vorstellen


\section{Loose Ends}
Warum sw frame?\\
Warum brauchen wir 3D? Um das zu entfalten und dann den Frame wechseln zu können.
Oder Überkapitel Solar Physics?
\\ \\
Heliosphere: Grenze zu LISM
\\ \\
Solar Wind: Zusammensetzung, schneller und langsamer, high latitudes: less complex, constant in speed \\ \\
B-Feldgleichung
\\ \\
Irgendwohin muss unbedingt Motivation, warum PUIs überhaupt interessant sind zu messen!
\\
PU Process\\ \\\
Instrument that is capable of measuring this distribution: large acceptance in absolute velocity, large variation and resolution in angles
%%%
%-------------------------------------------------------------
%%%
